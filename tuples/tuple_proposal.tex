\documentclass{article}

\usepackage[T1]{fontenc}
\usepackage[latin1]{inputenc}

%\usepackage[finnish]{babel}

\newcommand{\effects}{\noindent \textbf{Effects:}}
\newcommand{\returns}{\noindent \textbf{Returns:}}
\newcommand{\requires}{\noindent \textbf{Requires:}}
\newcommand{\notes}{\noindent \textbf{Notes:}}
\newcommand{\code}[1]{\texttt{#1}}
\author{Jaakko J�rvi}

\title{Proposal for adding tuple tuples into the standard library}


\begin{document}
\maketitle{}

\section{Text in the standard}

\subsection{Tuples}

The library provides a template for a heterogeneous list of objects, 
and an interface for easier handling of these lists as tuple types.

\noindent \textbf{Header} \verv|<tuple>| synopsis

\begin{verbatim}
class null_type {}

template <class H, class T> class cons; 

template <class T0 = null_type, class T1 = null_type, ..., 
          class TN = null_type> 
class tuple;

template<class T> int tuple_size(const T&);
int tuple_size(null_type);

template <int N, class T> class tuple_element;
template <int N, class H, class T> rt? get(cons<H, T>&);
template <int N, class H, class T> rt? get(const cons<H, T>&);

inline bool operator==(const null_type&, const null_type&);
inline bool operator!=(const null_type&, const null_type&);
inline bool operator<(const null_type&, const null_type&);
inline bool operator>(const null_type&, const null_type&);
inline bool operator>=(const null_type&, const null_type&);
inline bool operator<=(const null_type&, const null_type&);


template <class H1, class T1, class H2, class T2>
inline bool operator==(const cons<H1, T1>& a, const cons<H2, T2>& b);
template <class H1, class T1, class H2, class T2>
inline bool operator!=(const cons<H1, T1>& a, const cons<H2, T2>& b);
template <class H1, class T1, class H2, class T2>
inline bool operator<(const cons<H1, T1>& a, const cons<H2, T2>& b);
template <class H1, class T1, class H2, class T2>
inline bool operator>(const cons<H1, T1>& a, const cons<H2, T2>& b);
template <class H1, class T1, class H2, class T2>
inline bool operator<=(const cons<H1, T1>& a, const cons<H2, T2>& b);
template <class H1, class T1, class H2, class T2>
inline bool operator>=(const cons<H1, T1>& a, const cons<H2, T2>& b);

\end{verbatim}


The type of the heterogeneous list type is:

\begin{verbatim}
class null_type {};

template <class H, class T>
class cons {
  H head_;
  T tail_;
public:
  typedef H head_type;
  typedef T tail_type;

  head_type& head();
  const head_type& head() const;

  tail_type& tail();
  const tail_type& tail() const;

  cons();
  cons(const typename remove_volatile<head_type>::type& h,
       const tail_type& t);

  template <class H2, class T2>
  cons( const cons<H2, T2>& u ) : head(u.head), tail(u.tail) {}

  template <class HT2, class TT2>
  cons& operator=( const cons<HT2, TT2>& u );

  cons& operator=(const cons& u);

  template <class T1, class T2>
  cons& operator=( const std::pair<T1, T2>& u );
}
\end{verbatim}

\requires\ \code{T} is an instance of \code{cons}, or \code{T} is 
\code{null\_type}.


\begin{verbatim}
cons();
\end{verbatim}
\effects\ Default initializes \code{head\_} and \code{tail\_}.

\begin{verbatim}
cons(const typename remove_volatile<head_type>::type& h, 
     const tail_type& t);
\end{verbatim}
\effects\ Initializes \code{head\_} with \code{h}
and \code{tail\_} with \code{t}.\\
\notes\ 
Unless \code{head\_type} is a non-const reference type, it is allowed 
to call the constructor with a temporary as the first argument.
When binding temporaries to references, the reference must
be to a non-volatile and const type (8.5.3 (5)). Thus \code{volatile}
is removed and \code{const} added. Reference types stay intact.




\begin{verbatim}
template <class H2, class T2> cons(const cons<H2, T2>& u);
\end{verbatim}

\effects\ Initializes members from the corresponding members of the argument,
performing implicit conversions as needed.


\begin{verbatim}
template <class HT2, class TT2> cons& operator=(const cons<HT2, TT2>& u);
\end{verbatim}
\effects\ As if implemented: 
\code{*this.head\_ = u.head\_; *this.tail\_ = u.tail\_;}\\
\returns\ \code{*this}


\begin{verbatim}
cons& operator=(const cons& u);
\end{verbatim}
\effects\ As if implemented: 
\code{*this.head\_ = u.head\_; *this.tail\_ = u.tail\_;}\\
\returns\ \code{*this}\\
\notes\ Default assignment operator is illformed if a class contains
member variables of reference types (12.8 (12)). 
Therefore an explicit definition is needed.

\begin{verbatim}
template <class T1, class T2> cons& operator=(const std::pair<T1, T2>& u);
\end{verbatim}
\requires\ \code{tuple\_size<cons> == 2} (ref????).\\
\effects\ As if implemented: 
\code{*this.head\_ = u.first; *this.tail\_.head\_ = u.second;}\\
\returns\ \code{*this}

\begin{verbatim}
head_type& get_head();
\end{verbatim}
\returns\ Reference to \code{head\_}. 

\begin{verbatim}
const head_type& get_head() const;
\end{verbatim}
\returns\ Reference to \code{head\_}.\\
\notes\ Constness is shallow. If \code{head\_type} is a non-const reference 
type, returns a non-const reference.

\begin{verbatim}
tail_type& get_tail();
const tail_type& get_tail() const;
\end{verbatim}
\returns\ Reference to tail. %NIL CASE ???

The struct template
\begin{verbatim}
template<class T> class tuple_size;
\end{verbatim}
with an integral constant \code{value}.ksdjfksjflsjlfk

\requires\ \code{T} is convertible to an instance of \code{cons}.\\
\returns\ \code{tuple\_size(T::tail_type) + 1}.

\begin{verbatim}
int tuple_size(null_type);
\end{verbatim}
\returns\ \code{0}.

template <int N, class T> class tuple_element;
template <int N, class H, class T> rt? get(cons<H, T>&);
template <int N, class H, class T> rt? get(const cons<H, T>&);

inline bool operator==(const null_type&, const null_type&);
inline bool operator!=(const null_type&, const null_type&);
inline bool operator<(const null_type&, const null_type&);
inline bool operator>(const null_type&, const null_type&);
inline bool operator>=(const null_type&, const null_type&);
inline bool operator<=(const null_type&, const null_type&);


template <class H1, class T1, class H2, class T2>
inline bool operator==(const cons<H1, T1>& a, const cons<H2, T2>& b);
template <class H1, class T1, class H2, class T2>
inline bool operator!=(const cons<H1, T1>& a, const cons<H2, T2>& b);
template <class H1, class T1, class H2, class T2>
inline bool operator<(const cons<H1, T1>& a, const cons<H2, T2>& b);
template <class H1, class T1, class H2, class T2>
inline bool operator>(const cons<H1, T1>& a, const cons<H2, T2>& b);
template <class H1, class T1, class H2, class T2>
inline bool operator<=(const cons<H1, T1>& a, const cons<H2, T2>& b);
template <class H1, class T1, class H2, class T2>
inline bool operator>=(const cons<H1, T1>& a, const cons<H2, T2>& b);






%template <class T1, class T2, ... class T10>
%class tuple {
%public:


\end{document}

- Onko non-storeable trikki tarpeellinen

- Miksi assignment riitt�� paireista tupliin
- Mik� on tuple_size:n return type
