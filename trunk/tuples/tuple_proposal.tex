\documentclass{article}

\usepackage[T1]{fontenc}
\usepackage[latin1]{inputenc}

%\usepackage[finnish]{babel}

\author{Jaakko J�rvi}

\title{Proposal for adding tuple type into the standard library}

\begin{document}
\maketitle{}

\section{Text in the standard}

\subsection{Tuples}

The library provides a template for a heterogeneous list of objects, 
and an interface for easier handling of these lists as tuple types.

The type of the heterogeneous list type is:

template <class H, class T>
class {

  typedef typename 
    tuples::wrap_non_storeable_type<head_type>::type stored_head_type;

public:
  typedef H head_type;
  typedef T tail_type;

  stored_head_type head; 
  tail_type tail;
  
  typename tuples::access_traits_non_const<stored_head_type>::type 
  get_head();

  typename tuples::access_traits_non_const<tail_type>::type 
  get_tail();

  typename tuples::access_traits_const<stored_head_type>::type 
  get_head() const;
  
  typename tuples::access_traits_const<tail_type>::type 
  get_tail();

  cons();

  cons(typename tuples::access_traits_parameter<stored_head_type>::type& h,
       const tail_type& t);

  template <class H2, class T2>
  cons( const cons<H2, T2>& u ) : head(u.head), tail(u.tail) {}

  template <class HT2, class TT2>
  cons& operator=( const cons<HT2, TT2>& u ) { 
    head=u.head; tail=u.tail; return *this; 
  }

  // must define assignment operator explicitly, implicit version is 
  // illformed if HT is a reference (12.8. (12))
  cons& operator=(const cons& u) { 
    head = u.head; tail = u.tail;  return *this; 
  }

  template <class T1, class T2>
  cons& operator=( const std::pair<T1, T2>& u ) { 
    BOOST_STATIC_ASSERT(length<cons>::value == 2); // check length = 2
    head = u.first; tail.head = u.second; return *this;
  }


and an impleme

template <class T1, class T2, ... class T10>
class tuple {
public:


\end{document}

